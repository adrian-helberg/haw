\documentclass{beamer}
\usepackage[utf8]{inputenc}
\usepackage[english,german]{babel} 
\usepackage{listings} 
\usepackage{listings-golang}
\usepackage{tikz}

\usepackage[absolute,overlay]{textpos}
  \setlength{\TPHorizModule}{1mm}
  \setlength{\TPVertModule}{1mm}

\titlegraphic{\includegraphics[scale=0.3]{logoHaw.png}}
\title{
	\textit{Praktikum Programmieren} \\
	\textbf{\\Aufgabenblatt 1 - Entwurf} \\
	\scriptsize{B-AI2 PMP SS 2018}
}
\author{Adrian Helberg, Gruppe 2 \\\textbf{\\ Prüfer: Prof. Dr. Bernd Kahlbrandt}}
\date{\today}

\definecolor{mygreen}{rgb}{0,0.6,0}

\begin{document}
\lstset{
    frame=single,
    basicstyle=\footnotesize,
    keywordstyle=\color{blue},
    showstringspaces=false, 
    stringstyle=\color{mygreen},
    tabsize=4,
    language=Java
}

\maketitle

\frame{\tableofcontents}

\section{Interface Complex}
\begin{frame}
\frametitle{Interface Complex}

Das Interface \textit{Complex} soll als Basis f\"ur die Erstellung einer \textit{``immutable''} und einer \textit{``mutable''} Repr\"asentation
von komplexen Zahlen dienen.\\
Da Interfaces in Java 8 Default-Implementationen bietet, wird in diesem Projekt nicht auf eine abstrakte Klasse Complex zur\"uckgegriffen,
da so viele Zeilen Code gespart werden k\"onnen.

\end{frame}

\begin{frame}
\frametitle{Interface Complex}

Das Interface wird wie folgt zusammengesetzt:

\begin{itemize}
\setlength{\itemsep}{12pt}
\item Diverse abstrakte Funktionssignaturen, die von erbenden Klassen \"uberschrieben werden m\"ussen
\item Default-Implementationen f\"ur \textit{``immutable''} Objekte, die nur von \textit{mutable} Objekten \"uberschrieben werden m\"ussen,
da diese eine andere Implementation brauchen
\item Dokumentation in Form von \textbf{Javadoc}
\end{itemize}

\end{frame}

\section{Klasse ImmutableComplex}
\begin{frame}
\frametitle{Klasse ImmutableComplex}

Die Klasse \textit{ImmutableComplex} erzeugt nicht ver\"anderbare (\textbf{final}) Objekte als komplexe Zahl.
Die Instanzmethoden geben immer ein neuen Objekt zur\"uck, wenn bestimmte Operationen auf den Objekten ausgef\"uhrt werden.\\

Die Klasse implementiert das Interface \textit{Complex} und besteht aus
\begin{itemize}
\setlength{\itemsep}{12pt}
\item Diversen Konstruktoren
\item \"Uberschriebenen Instanzmethoden
\item Allen \"ubrigen Instanzmethoden, die eine Default-Implementation im zu erbenden Interface besitzen
\end{itemize}

\end{frame}

\section{Klasse MutableComplex}
\begin{frame}
\frametitle{Klasse MutableComplex}

Die Klasse \textit{MutableComplex} erzeugt ver\"anderbare Objekte als komplexe Zahl.
Die Instanzmethoden ver\"andern die aktuelle Instanz (\textbf{this}) und geben, wenn bestimmte Operationen auf den Objekten ausgef\"uhrt werden,
diese anschließend zur\"uck.\\

Die Klasse implementiert das Interface \textit{Complex} und besteht aus
\begin{itemize}
\setlength{\itemsep}{12pt}
\item Diversen Konstruktoren
\item \"Uberschriebenen Instanzmethoden
\item Allen \"ubrigen Instanzmethoden, die nicht \"uberschrieben wurden und eine Default-Implementation im zu erbenden Interface besitzen
\end{itemize}

\end{frame}

\section{Klasse MathUtils}
\begin{frame}
\frametitle{Klasse MathUtils}

Die Klasse \textit{MathUtils} stellt diverse statische Funktionen f\"ur komplexe Zahlen zur Verf\"ugung.
Unter anderem aber auch eine \textit{round}-Funktion, um auf eine bestimmte Anzahl Nachkommastellen zu runden.
Da es sich bei den komplexen Zahlen um ver\"anderbare und nicht ver\"anderbare Objekte handelt, wurden
einige Funktionen mit dem \textit{Reflection-Tool} \textbf{instance of} implementiert.

\end{frame}

\section{JUnit Tests}
\begin{frame}
\frametitle{JUnit Tests}

Alle implementierten Konstruktoren, Funktionen und Methoden werden mit \textit{JUnit} \textbf{assertions} in den Test-Klassen
\begin{itemize}
\item \textit{ImmutableComplexTest}
\item \textit{MutableComplexTest}
\item \textit{MathUtilsTest}
\end{itemize}
getestet.

\end{frame}

\section{Instanz-Methoden}
\begin{frame}
\frametitle{Instanz-Methoden}

Folgende Instanz-Methoden werden implementiert \footnote{siehe https://ruby-doc.org/core-2.3.0/Complex.html}
\begin{itemize}
\item \textit{getRe}
\item \textit{getIm}
\item \textit{getR}
\item \textit{getTheta}
\item \textit{multiply}
\item \textit{add}
\item \textit{subtract}
\item \textit{reciprocal}
\item \textit{negate}
\end{itemize}

\end{frame}

\begin{frame}
\frametitle{Instanz-Methoden}

\begin{itemize}
\item \textit{divide}
\item \textit{equals}
\item \textit{abs}
\item \textit{abs2}
\item \textit{conjugate}
\item \textit{hash}
\item \textit{formatCartesian}
\item \textit{formatTrigonometric}
\item \textit{formatPolar}
\end{itemize}

\end{frame}

\end{document}