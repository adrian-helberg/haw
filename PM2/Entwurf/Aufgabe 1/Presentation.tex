\documentclass{beamer}
\usepackage[utf8]{inputenc}
\usepackage[english,german]{babel} 
\usepackage{listings} 
\usepackage{listings-golang}
\usepackage[absolute,overlay]{textpos}
\usepackage{tcolorbox}
\usepackage{multicol}
\usepackage[absolute,overlay]{textpos}
  \setlength{\TPHorizModule}{1mm}
  \setlength{\TPVertModule}{1mm}

\setbeamertemplate{footline}[frame number]

\titlegraphic{
\includegraphics[scale=0.1]{golanglogo.png}
\begin{textblock}{36}(0,80)
\includegraphics[scale=0.4]{logoHaw.png}
\end{textblock}
}
\title{
	\Large{\textit{\\Die Programmiersprache Go - Eine Einf\"uhrung}} \\
	\Large{\textbf{\\Seminarvortrag}}
}
\author{Student: Adrian Helberg \\Prüfer: Prof. Dr. Axel Schmolitzky}
\date{\today}

\definecolor{mygreen}{rgb}{0,0.6,0}

\begin{document}
\lstset{
    frame=single,
    basicstyle=\scriptsize,
    keywordstyle=\color{blue},
    showstringspaces=false, 
    stringstyle=\color{mygreen},
    tabsize=4,
    language=Golang
}

\maketitle

\frame{
\frametitle{Inhalt}
\tableofcontents[hideothersubsections]
}

\begin{frame}

\begin{quote}
``Go is an open source programming language that makes it easy to build simple, reliable and efficient software." 
\end{quote}

\begin{flushright}
\scriptsize (Go Website: \href{golang.org}{golang.org})
\end{flushright}

\begin{quote}
Go ist eine Open-Source-Programmiersprache, die es einfach macht, simple, zuverlässige und effiziente Software zu erstellen.
\end{quote}

\begin{flushright}
\scriptsize (Eigene \"Ubersetzung)
\end{flushright}

\end{frame}

%%%%%%%%%%%%%%%%%% GESCHICHTE %%%%%%%%%%%%%%%%%%

\section{Geschichte}
\begin{frame}
\frametitle{Geschichte}

\centering
\begin{figure}
\includegraphics[scale=0.5]{designers.png}
\caption{Robert Griesemer, Rob Pike und Ken Thompson.}
\end{figure}

\end{frame}

%%%%%%%%%%%%%%%%%% ENTWICKLER %%%%%%%%%%%%%%%%%%
\subsection{Entwickler}
\begin{frame}
\frametitle{Entwickler}

\begin{itemize}
\setlength{\itemsep}{24pt}
\item Konzipiert September 2007
\item Robert Griesemer, Rob Pike und Ken Thompson
\item Mitarbeiter von Google LLC \textregistered
\item Aus Frust heraus entstanden
\item \textit{``Complexity is multiplicative''} - Rob Pike
\end{itemize}

\end{frame}

%%%%%%%%%%%%%%%%%% ENTWURFSPHASE %%%%%%%%%%%%%%%%%%

\subsection{Entwurfsphase}
\begin{frame}
\frametitle{Entwurfsphase}

\begin{itemize}
\setlength{\itemsep}{20pt}
\item Ausdrucksstarke und effiziente Kombination aus Kompilierung und Ausf\"uhrung
\item Ähnlichkeiten mit C
\item Adaptiert gute Ideen aus einigen Programmiersprachen: \\
Pascal, Modula-2, Oberon, Oberon-2, Alef, ...
\end{itemize}

\end{frame}

\begin{frame}
\frametitle{Entwurfsphase}

\begin{figure}
\centering
\includegraphics[scale=0.45]{origin.png}
\caption{The Go Programming Language,  Preface xii}
\end{figure}

\end{frame}

\begin{frame}
\frametitle{Entwurfsphase}

\begin{itemize}
\setlength{\itemsep}{40pt}
\item Vermeiden von Features, die zu komplexem, unzuverl\"assigem Code führen w\"urden
\item M\"oglichkeiten zur Nebenl\"aufigkeit sind neu und effizient
\item Datenabstraktion und Objektorientierung sind ungewohnt flexibel
\item Automatische Speicherverwaltung (garbage collection)
\end{itemize}

\end{frame}

%%%%%%%%%%%%%%%%%% VERÖFFENTLICHUNG %%%%%%%%%%%%%%%%%%

\subsection{Ver\"offentlichung}
\begin{frame}
\frametitle{Ver\"offentlichung}

\begin{itemize}
\setlength{\itemsep}{40pt}
\item Vorgestellt November 2009
\item Stable Release am 16. Februar 2018
\item Ber\"uhmt als Nachfolger für nicht typisierte Scriptsprachen
\begin{itemize}
\item[] $\rightarrow$ Verbindung aus Ausdruckskraft und Sicherheit
\end{itemize}
\end{itemize}

\end{frame}

%%%%%%%%%%%%%%%%%% GO-COMMUNITY %%%%%%%%%%%%%%%%%%

\subsection{Go-Community}
\begin{frame}
\frametitle{Go-Community}

\begin{itemize}
\setlength{\itemsep}{44pt}
\item Open-source projekt
\begin{itemize}
\item[] $\rightarrow$ Quellcode des Compilers, Bibliotheken (libraries) und Tools sind frei verfügbar
\end{itemize}
\item Aktive, weltweite Community
\item L\"auft auf Unix, Mac und Windows
\begin{itemize}
\item[] $\rightarrow$ \"Ublicherweise ohne Modifikation transportierbar
\end{itemize}
\end{itemize}

\end{frame}

\begin{frame}
 %\frametitle{Statistik}

\begin{figure}
\includegraphics[scale=0.3]{graph.png}
\caption{Go im Vergleich}
\end{figure}

\end{frame}

%%%%%%%%%%%%%%%%%% MERKMALE %%%%%%%%%%%%%%%%%%

\section{Merkmale}
\begin{frame}
\frametitle{Merkmale}

\begin{textblock}{50}(40, 57)
\begin{figure}
\caption{Golang ``Gopher''}
\includegraphics[scale=0.35]{gopherHead.png}
\end{figure}
\end{textblock}

\end{frame}

%%%%%%%%%%%%%%%%%% HELLO WORLD %%%%%%%%%%%%%%%%%%

\begin{frame}[fragile]
\frametitle{Hello World!}

Go
\begin{lstlisting}
package main
import "fmt"

func main() {
    fmt.Println("Hello World!")
}
\end{lstlisting}

Java
\lstset{language=Java}
\begin{lstlisting}
public class HelloWorld 
{ 
    public static void main (String[] args)
    {
        System.out.println("Hello World!");
    }
}
\end{lstlisting}

\begin{textblock}{32}(86,41)
\begin{tcolorbox}
\textit{Ausgabe:\\}\\
Hello World!
\end{tcolorbox}
\end{textblock}

\end{frame}

%%%%%%%%%%%%%%%%%% TYPSICHERHEIT %%%%%%%%%%%%%%%%%%

\subsection{Typsicherheit}
\begin{frame}
\frametitle{Typsicherheit}

\begin{itemize}
\setlength{\itemsep}{24pt}
\item Statische Typisierung
\item Features simulieren dynamische Typisierung
\begin{itemize}
\setlength{\itemsep}{12pt}
\item Keine explizite Markierung von Interface-Implementierungen (Java: \textit{implements})
\item Stimmt eine Methodensignatur mit der des Interfaces \"uberein, wird diese automatisch implementiert (\"ahnlich: \textit{duck-typing})
\item Einfaches Erweitern externer Methoden (Library Funtionen)
\end{itemize}
\item OOP durch \textit{struct} und \textit{interface} m\"oglich
\end{itemize}

\end{frame}

%%%%%%%%%%%%%%%%%% OBJEKTORIENTIERUNG %%%%%%%%%%%%%%%%%%

\subsection{Objektorientierung}
\begin{frame}[fragile]
\frametitle{Objektorientierung}

\begin{itemize}
\item Typ \textit{struct}
\begin{itemize}
\setlength{\itemsep}{24pt}
\item Sammlung von Variablen und Funktionen
\item Methoden nicht virtuell
\item Man spricht bei Funktionen von \textbf{Methoden} durch Zugeh\"origkeit des \textit{structs}
\item Konvention: Nutzen von Zeigern bei ``Settern'',\\ \textit{Wertkopien} bei ``Gettern''
\end{itemize}
\end{itemize}

\end{frame}

\begin{frame}
\frametitle{Objektorientierung}

\begin{itemize}
\item Typ \textit{interface}
\begin{itemize}
\setlength{\itemsep}{24pt}
\item Sammlung von Funktionssignaturen
\item Objekt vom Typ \textit{Circle} kann einer Variable vom Typ \textit{Shape} zugeordnet werden, weil \textit{Circle} die notwendige Funktion \textbf{Area} bietet
\item Mehrfachvererbung m\"oglich!
\end{itemize}
\end{itemize}

\end{frame}

\begin{frame}[fragile]
\frametitle{Objektorientierung}

\begin{lstlisting}
type Shape interface {
    Area() float64
}

type Circle struct {
    radius float64
}

func (c Circle) Area() float64 {
	return math.Pi * c.radius * c.radius
}

func main() {
    var shape Shape = Circle{2}
    fmt.Println(shape.Area())
}
\end{lstlisting}

\begin{textblock}{42}(80,60)
\begin{tcolorbox}
\textit{Ausgabe:\\}\\
12.566370614359172
\end{tcolorbox}
\end{textblock}

\end{frame}

\begin{frame}[fragile]
\frametitle{Objektorientierung}

\begin{itemize}
\item Polymorphie
\end{itemize}

\begin{lstlisting}
type Shape interface {
  Area() float64
}

func (c Circle) Area() float64{
  return math.Pi * c.radius * c.radius
}

func (r Rectangle) Area() float64{
  return r.length * r.width
}
\end{lstlisting}

\begin{itemize}
\setlength{\itemsep}{24pt}
\item Kein ``echtes'' \textit{Duck-Typing}, da statische Typ\"uberpr\"ufung
\end{itemize}

\end{frame}

%%%%%%%%%%%%%%%%%% SPEICHERBEREINIGUNG %%%%%%%%%%%%%%%%%%

\subsection{Speicherbereinigung}
\begin{frame}
\frametitle{Speicherbereinigung}

\begin{itemize}
\setlength{\itemsep}{30pt}
\item Automatisch
\item \textit{Garbage Collector}
\item Wird eine Variable unerreichbar, wird ihr Speicherbereich freigegeben
\end{itemize}

\end{frame}

%%%%%%%%%%%%%%%%%% ERROR HANDLING %%%%%%%%%%%%%%%%%%

\section{Error Handling}
\begin{frame}[fragile]
\frametitle{Error Handling}

\begin{lstlisting} 
func main() {

    f, err := os.Open("filename.ext")

    if err != nil {
        log.Fatal(err)
    }

    defer f.Close()
}
\end{lstlisting}

\begin{itemize}
\item Panik-System
\item \textit{defer} statt \textit{finally}
\item Benutzerdefinierte Errors: \textit{errors.New(``oh no'')}
\end{itemize}

\end{frame}

\begin{frame}[fragile]
\frametitle{Error Handling}

\begin{lstlisting} 
func main() {

    f , _ := os.Open("filename.ext")

    defer f.Close()
}
\end{lstlisting}

\end{frame}

%%%%%%%%%%%%%%%%%% PERFORMANCE %%%%%%%%%%%%%%%%%%

\section{Performance}
\begin{frame}
\frametitle{Performance}

\centering
\begin{figure}
\includegraphics[scale=0.6]{performance1.png}
\caption{64-bit Ubuntu quad core}
\end{figure}

\end{frame}

\begin{frame}
\frametitle{Performance}

\centering
\begin{figure}
\includegraphics[scale=0.6]{performance2.png}
\caption{64-bit Ubuntu quad core}
\end{figure}

\end{frame}

%%%%%%%%%%%%%%%%%% SPRACHMITTEL %%%%%%%%%%%%%%%%%%

\section{Sprachmittel}
\begin{frame}
\frametitle{Sprachmittel}

\centering
\begin{figure}
\includegraphics[scale=0.45]{sprachmittel.png}
\caption{Noch ein ``Gopher''!}
\end{figure}

\end{frame}

%%%%%%%%%%%%%%%%%% CLOSURES %%%%%%%%%%%%%%%%%%

\subsection{Closures}
\begin{frame}[fragile]
\frametitle{Closures}

Java
\lstset{language=Java}
\begin{lstlisting}
private static Function<String, Supplier<String>> intSeq = 
x -> {

    AtomicInteger atomicInteger = new AtomicInteger();
    return () -> x + ": " + atomicInteger.incrementAndGet();
}

public static void main(String[] args) {

    Supplier<String> nextInt = intSeq.apply("Test 1");

    System.out.println(nextInt.get());
    System.out.println(nextInt.get());
}
\end{lstlisting}

\begin{textblock}{36}(88,66)
\begin{tcolorbox}
\textit{Ausgabe:\\}\\
Test 1: 1 \\
Test 1: 2
\end{tcolorbox}
\end{textblock}

\end{frame}

\begin{frame}[fragile]
\frametitle{Closures}
Go
\begin{lstlisting}
func intSeq(x string) func() string {
	i := 0
	return func() string {
		i++
		return x + ": " + strconv.Itoa(i)
	}
}

func main() {
	nextInt := intSeq("Test 1")

	fmt.Println(nextInt())
	fmt.Println(nextInt())
}
\end{lstlisting}

\begin{textblock}{36}(88,66)
\begin{tcolorbox}
\textit{Ausgabe:\\}\\
Test 1: 1 \\
Test 1: 2
\end{tcolorbox}
\end{textblock}

\end{frame}

%%%%%%%%%%%%%%%%%% REFLECTION %%%%%%%%%%%%%%%%%%

\subsection{Reflection}
\begin{frame}[fragile]
\frametitle{Reflection}

Java
\lstset{language=Java}
\begin{lstlisting}
public static String getStringProperty(Object object, 
                                        String methodname) {
    String value = null;

    try {
        Method getter = object.getClass()
                        .getMethod(methodname, new Class[0]);

        value = (String) getter
                        .invoke(object, new Object[0]);

    } catch (Exception e) {}

    return value;
}
\end{lstlisting}

\end{frame}

\begin{frame}[fragile]
\frametitle{Reflection}

Go
\begin{lstlisting}
func getField(v *Vertex, field string) int {

    r := reflect.ValueOf(v)
    f := reflect.Indirect(r).FieldByName(field)

    return int(f.Int())
}
\end{lstlisting}

\end{frame}

\begin{frame}
\frametitle{Reflection}

\begin{itemize}
\setlength{\itemsep}{24pt}
\item func ValueOf(i interface{}) Value
\begin{itemize}
\item Gibt ein Objekt Value (reflection interface) zurück, das auf den konkreten Wert initialisiert wurde, der in der Schnittstelle i gespeichert ist
\end{itemize}
\item func Indirect(v Value) Value
\begin{itemize}
\item Gibt den Wert zurück, auf den v zeigt
\end{itemize}
\item func (Value) FieldByName
\begin{itemize}
\item Gibt das \textit{struct field} mit dem angegebenen Namen zur\"uck
\end{itemize}
\item func (v Value) Int() int64
\begin{itemize}
\item Gibt den zugrunde liegenden Wert von v zur\"uck
\end{itemize}
\end{itemize}

\end{frame}

%%%%%%%%%%%%%%%%%% GOROUTINES %%%%%%%%%%%%%%%%%%

\subsection{Goroutines}

\begin{frame}
\frametitle{Goroutines}

\begin{itemize}
\setlength{\itemsep}{24pt}
\item Philosophie: \textit{``Kommuniziere nicht, indem du Speicher teilst, sondern teile Speicher durch Kommunikation'}
\item Keine Einschr\"ankung beim Nutzen unsicherer Zugriffsmethode
\item \"Ublich: Goroutines, Channels
\begin{itemize}
\item Keine ``Race Conditions''
\end{itemize}
\end{itemize}

\end{frame}

\begin{frame}[fragile]
\frametitle{Goroutines}

\begin{itemize}
\setlength{\itemsep}{24pt}
\item Schl\"usselwort \textit{go}
\item Kommunikation \"uber \textit{channels}:
\end{itemize}

\begin{lstlisting}
messages := make(chan string)

go func() { messages <- "ping" }()
\end{lstlisting}

\begin{lstlisting}
msg := <- messages

fmt.Println(msg)
\end{lstlisting}

\end{frame}

%%%%%%%%%%%%%%%%%% DOCKER %%%%%%%%%%%%%%%%%%

\section{Docker}
\begin{frame}
\frametitle{Docker}

\begin{itemize}
\setlength{\itemsep}{12pt}
\item Open-Source Apache 2.0 Lizenz
\item Basiert auf \textit{Namespaces}
\item Isolierung von Anwendungen mit Containervirtualisierung
\item Ressourcentrennung (Code, Laufzeitmodul, Systemwerkzeuge, Systembibliotheken, ...)
\item Erstellung von Containern mit virtuellen Betriebssystemen
\end{itemize}

\begin{textblock}{20}(84,0)
\includegraphics[scale=0.22]{docker.png}
\end{textblock}

\end{frame}

\begin{frame}
\frametitle{Docker}

\begin{itemize}
\setlength{\itemsep}{20pt}
\item Docker Hub (Online-Dienst) als Verteiler fest integriert
\item Eingebaute Versionsverwaltung, angelehnt an
\end{itemize}

\begin{textblock}{20}(84,0)
\includegraphics[scale=0.22]{docker.png}
\end{textblock}

\begin{textblock}{20}(92,49)
\includegraphics[scale=0.22]{gitlogo.png}
\end{textblock}

\end{frame}

%%%%%%%%%%%%%%%%%% ZENTRALE FRAGESTELLUNG %%%%%%%%%%%%%%%%%%

\begin{frame}
\frametitle{Zentrale Fragestellung}

\centering
\glqq flexibel wie dynamisch getyped,\\ aber mit statischer Typsicherheit?\grqq{}

\end{frame}

%%%%%%%%%%%%%%%%%% PRO & CONTRA %%%%%%%%%%%%%%%%%%

\section{Pro \& Contra}
\begin{frame}
\frametitle{Pro \& Contra}

\begin{tabular}{p{5cm} | p{5.5cm}}
\textbf{Pro} & \textbf{Contra} \\ \hline
\begin{itemize}
\setlength{\itemsep}{20pt}
\item Minimalismus
\item ``Eigenes'' Duck-Typing
\item Parallelisierung
\item Aufger\"aumte Syntax
\item Schneller Compiler
\end{itemize}
&
\begin{itemize}
\setlength{\itemsep}{20pt}
\item Keine generische Programmierung
\item nil statt Option
\item Wenig grundlegende Datenstrukturen
\item Keine Methodenüberladung
\end{itemize}
\\
\end{tabular}

\end{frame}

\begin{frame}
\frametitle{Pro \& Contra}

\begin{tabular}{p{5cm} | p{5.5cm}}
\textbf{Pro} & \textbf{Contra} \\ \hline
\begin{itemize}
\setlength{\itemsep}{20pt}
\item Statisch gelinkte Binärdateien
\item Laufzeiteigenschaften
\item Integriertes Unit-Test-Framework
\item Paketmanager
\end{itemize}
&
\begin{itemize}
\setlength{\itemsep}{20pt}
\item Unbefriedigende API-Dokumentation
\item Teilweise umständliche APIs
\item Umst\"andliches Mocking
\end{itemize}
\\
\end{tabular}

\end{frame}

\section{Einzelnachweise}
\begin{frame}
\frametitle{Einzelnachweise}

\begin{figure}
\includegraphics[scale=0.3]{book.png}
\caption{The Go Programming Language, von Alan A. A. Donovan und Brian W. Kernighan, 2016}
\end{figure}

\end{frame}

\begin{frame}
\frametitle{Einzelnachweise}

\begin{itemize}
\setlength{\itemsep}{20pt}
\item Performance-Grafik: \href{https://benchmarksgame.alioth.debian.org/u64q/go.htmll}{benchmarksgame.alioth.debian.org/u64q/go.html}
\item Diverse ``Gopher'' Grafiken: \href{https://golang.org/}{golang.org}
\item Bild der Entwickler, Bild ``Go im Vergleich'': \href{https://www.quora.com/Why-should-I-learn-Go-Golang-instead-of-Scala-Kotlin-Rust-Erlang-Haskell-Clojure-OCaml-etc}{quora.com/Why-should-I-learn-Go-Golang-instead-of-Scala-Kotlin-Rust-Erlang-Haskell-Clojure-OCaml-etc}
\item Bild ``Docker'': \href{https://www.docker.com/}{docker.com}
\end{itemize}

\end{frame}

\begin{frame}
\frametitle{Ende}

\centering
Danke f\"ur die Aufmerksamkeit!

\begin{figure}
\includegraphics[scale=0.4]{joda.png}
\end{figure}

\end{frame}

\end{document}