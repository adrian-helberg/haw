\documentclass{beamer}
\usepackage[utf8]{inputenc}
\usepackage[english,german]{babel} 
\usepackage{listings} 
\usepackage{listings-golang}
\usepackage{tikz}
\usepackage{hyperref}

\usepackage[absolute,overlay]{textpos}
  \setlength{\TPHorizModule}{1mm}
  \setlength{\TPVertModule}{1mm}

\titlegraphic{\includegraphics[scale=0.3]{logoHaw.png}}
\title{
	\textit{Praktikum Programmieren} \\
	\textbf{\\Aufgabenblatt 4 - Entwurf} \\
	\scriptsize{B-AI2 PMP SS 2018}
}
\author{Adrian Helberg, Rodrigo Ehlers , Gruppe 2 \\\textbf{\\ Prüfer: Prof. Dr. Bernd Kahlbrandt}}
\date{\today}

\definecolor{mygreen}{rgb}{0,0.6,0}

\begin{document}
\lstset{
    frame=single,
    basicstyle=\footnotesize,
    keywordstyle=\color{blue},
    showstringspaces=false, 
    stringstyle=\color{mygreen},
    tabsize=4,
    language=Java
}

\maketitle

\frame{\tableofcontents}

%%%%%%%%%% AUFGABE 1 %%%%%%%%%%

\section{Collections}
\begin{frame}[fragile]
\frametitle{Collections - Interface Deque}

\begin{itemize}
\setlength{\itemsep}{8pt}
\item Stack-artig
\end{itemize}

\begin{lstlisting}
/**
 * Pushes an item onto the top of the stack
 * @param item the item to be pushed onto this stack
 */
void push(E item);

/**
 * Removes the object at the top of this stack and
 * returns that object
 * @throws EmptyStackException
 */
void pop() throws EmptyStackException;
\end{lstlisting}

\end{frame}

\begin{frame}[fragile]
\frametitle{Collections - Interface Deque}

\begin{lstlisting}
/**
 * Looks at the object at the bottom of this stack 
 * without removing it from the stack
 * @return the object at the bottom of this stack
 * @throws NoSuchElementException if this queue is 
 * empty
 */
E peekLast() throws NoSuchElementException;
\end{lstlisting}

\end{frame}

\begin{frame}[fragile]
\frametitle{Collections - Interface Deque}

\begin{itemize}
\setlength{\itemsep}{8pt}
\item Queue-artig
\end{itemize}

\begin{lstlisting}
/**
 * Enqueue the item
 * @param item to add
 * @throws NullPointerException if the specified item 
 * is null
 */
void enqueue(E item) throws NullPointerException;

/**
 * Dequeue the item
 * @throws NoSuchElementException
 */
void dequeue() throws NoSuchElementException;
\end{lstlisting}

\end{frame}

\begin{frame}[fragile]
\frametitle{Collections - Interface Deque}

\begin{lstlisting}
/**
 * Looks at the object at the top of this stack without 
 * removing it from the stack
 * @return the object at the top of this stack
 * @throws NoSuchElementException if this queue is 
 * empty
 */
E peekFirst() throws NoSuchElementException;
\end{lstlisting}

\end{frame}

\begin{frame}[fragile]
\frametitle{Collections - Interface Deque}

\begin{itemize}
\setlength{\itemsep}{8pt}
\item Methode isEmpty()
\end{itemize}

\begin{lstlisting}
/**
 * Tests if this stack is empty
 * @return true if and only if this stack contains 
 * no items; false otherwise
 */
boolean isEmpty();
\end{lstlisting}

\end{frame}

\begin{frame}[fragile]
\frametitle{Collections - Interface Deque}

\begin{itemize}
\setlength{\itemsep}{8pt}
\item NULL-Handling
\end{itemize}

\begin{lstlisting}
/**
 * Checks for NULL; does not allow NULL
 * @param data data to check
 * @throws NullPointerException if data is null
 */
default void checkNull(E data) 
                          throws NullPointerException {
    if (data == null) {
        throw new NullPointerException(
            "Null not allowed"
        );
    }
}
\end{lstlisting}

\end{frame}

\section{Denksportaufgaben}
\begin{frame}[fragile]
\frametitle{Denksportaufgaben}

1. Geben Sie bitte Deklarationen für die Variablen x und i an, für die

\begin{lstlisting}
x += i;
\end{lstlisting}

lagal ist, aber

\begin{lstlisting}
x = x + i;
\end{lstlisting}

nicht. \\
\begin{lstlisting}
int x = 1;
long y = 1;
x += y;
\end{lstlisting}

\end{frame}

\begin{frame}
\frametitle{Denksportaufgaben}

\begin{quote}
Finden Sie bitte die Begründung in der JLS, warum das zweite Konstrukt nun funktioniert
und erklären Sie dies!
\end{quote}

\textit{5.4. String Conversion}\\

\textbf{
\\
String conversion applies only to an operand of the binary + operator which is not a String when the other operand is a String.\\
In this single special case, the non-String operand to the + is converted to a String (§5.1.11) and evaluation of the + operator proceeds as specified in §15.18.1.
}


\end{frame}

\begin{frame}[fragile]
\frametitle{Denksportaufgaben}

Schreiben Sie bitte eine Klasse A, so dass der Konstruktor der folgenden Klasse B
„Win“ ausgibt!

\begin{lstlisting}
[...]
new B(Long.MAX_VALUE);

class A {
    A Long = this;

    int compare(Long a, Long b) {
        return 1;
    }

    class B {
        B (Long i) {}
    }
}
\end{lstlisting}

\end{frame}

\end{document}