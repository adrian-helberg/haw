\documentclass{beamer}
\usepackage[utf8]{inputenc}
\usepackage[english,german]{babel} 
\usepackage{listings} 
\usepackage{listings-golang}
\usepackage{tikz}
\usepackage{hyperref}

\usepackage[absolute,overlay]{textpos}
  \setlength{\TPHorizModule}{1mm}
  \setlength{\TPVertModule}{1mm}

\titlegraphic{\includegraphics[scale=0.3]{logoHaw.png}}
\title{
	\textit{Praktikum Programmieren} \\
	\textbf{\\Aufgabenblatt 2 - Entwurf} \\
	\scriptsize{B-AI2 PMP SS 2018}
}
\author{Adrian Helberg, Gruppe 2 \\\textbf{\\ Prüfer: Prof. Dr. Bernd Kahlbrandt}}
\date{\today}

\definecolor{mygreen}{rgb}{0,0.6,0}

\begin{document}
\lstset{
    frame=single,
    basicstyle=\scriptsize,
    keywordstyle=\color{blue},
    showstringspaces=false, 
    stringstyle=\color{mygreen},
    tabsize=4,
    language=Java
}

\maketitle

\frame{\tableofcontents}

\section{Listen und IO}
\begin{frame}
\frametitle{Listen und IO}

Um die 1000 komplexen Zahlen nach vorgegebenen Regeln zu erstellen und in eine Textdatei zu schreiben gibt es folgende Strukturen:

\begin{itemize}
\item Einen \textit{PrintWriter}, der den Namen der Textdatei entgegennimmt
\item Ein Array der Gr\"oße 1000, welches die komplexen Zahlen h\"alt
\item Eine \textit{for}-Kontrollstruktur, welche das deklarierte Array iteriert und mit komplexen Zahlen initialisiert
\item Eine Anweisung \textit{writer.println(...)}, welche die angegebene Datei zeilenweise beschreibt
\end{itemize}

\end{frame}

\begin{frame}
\frametitle{Listen und IO}

Um komplexe Zahlen aus einer Textdatei zu lesen und  zu speichern, gibt es folgende Strukturen:

\begin{itemize}
\item Einen \textit{BufferedReader}, der die angegebene Textdatei liest
\item Ein Array der Gr\"oße 1000, welches die komplexen Zahlen h\"alt
\item Eine Liste, welche die komplexen Zahlen h\"alt
\item Eine \textit{while}-Kontrollstruktur, welche die Textdatei zeilenweise durchl\"auft
\item Zwei Zuweisungen, welche die resultierenden, komplexen Zahlen in das Array und die Liste schreiben
\end{itemize}

\end{frame}

\begin{frame}[fragile]
\frametitle{Listen und IO}

Das Sortieren des Arrays bzw. der Liste wird \"uber die Hilfsklasse \textit{MathUtils} umgesetzt:

\begin{lstlisting}
public static Complex[] sortByLength(Complex[] n);

public static List<Complex> sortByLength(List<Complex> n);
\end{lstlisting}

Um sie Sortierung zu erm\"oglichen wird eine interne Klasse \textit{Tupel} verwendet, welche die Funktion \textit{compareTo(...)} des
Interfaces \textit{Comparable$<$Tuple$>$} \"uberschreibt:

\begin{lstlisting}
public int compareTo(Tuple t) {
	return this.length < t.length 
		? -1 : (this.length > t.length 
			? 1 : 0);
}
\end{lstlisting}

\end{frame}

\section{Operatoren}
\begin{frame}[fragile]
\frametitle{Operatoren}

\begin{lstlisting}
public static boolean isOdd(long i) {
	return i % 2 == 1;
}
\end{lstlisting}

\begin{itemize}
\item Ist die Methode korrekt?
	\begin{itemize}
	\item \textit{ Nein}
	\end{itemize}
\item Wenn ja, warum, wenn nicht, warum?
	\begin{itemize}
	\item \textit{Sollte eine negative Zahl \"uberpr\"uft werden, liefert die Funktion ein falsches Ergebnis}\footnote{Das Ergebnis der Restoperation ist negativ, wenn die Division negativ, und positiv, wenn die Division positiv ist}
	\end{itemize}
\item Was liefert sie und wie macht man das richtig?
	\begin{itemize}
	\item Die Funktion liefert f\"ur \textit{isOdd(-1L)} den Wert \textit{false}.
	\end{itemize}
\end{itemize}

\begin{lstlisting}
public static boolean isOdd(long i) {
	return (i % 2 + 2) % 2 == 1;
}
\end{lstlisting}

\end{frame}

\begin{frame}[fragile]
\frametitle{Operatoren}

\begin{lstlisting}
System.out.println((int) (char) (byte) -1);
\end{lstlisting}

\begin{itemize}
\item Welches Ergebnis erwarten Sie?
	\begin{itemize}
	\item \textit{Erwartet wird hier -1}
	\end{itemize}
\item Welches kommt tats\"achlich heraus?
	\begin{itemize}
	\item \textit{Der Ausdruck liefert 65535}
	\end{itemize}
\item Was passiert genau?
	\begin{itemize}
	\item \textit{Zuerst wird das Byte ``-1'' via widening primitive conversion (§5.1.2), und dann der resultierende int via narrowing primitive conversion (§5.1.3) konvertiert. Anschließend wird das resultierende ``?'' in int konvertiert und ergibt 65535, da ``?'' der Hexzahl ``FFFFFFFF'' entspricht und damit des Maximalwerts des 16 bit integers}
	\item siehe \textit{Widening and Narrowing Primitive Conversion}\footnote{\href{https://docs.oracle.com/javase/specs/jls/se8/html/jls-5.html}{https://docs.oracle.com/javase/specs/jls/se8/html/jls-5.html}}
	\end{itemize}
\end{itemize}

\end{frame}

\begin{frame}[fragile]
\frametitle{Operatoren}

\begin{lstlisting}
int x = 1984;
int y = 2001;
x ^= y ^= x ^ y;
System.out.println("x = " + x + "; y = " + y);
\end{lstlisting}

\begin{itemize}
\item Was wird ausgegeben?
	\begin{itemize}
	\item \textit{x = 0; y = 1984}
	\item Der Ausdruck wird von rechts nach links evaluiert
	\end{itemize}
\end{itemize}

\end{frame}

\begin{frame}[fragile]
\frametitle{Operatoren}

Schritt 1:

\begin{lstlisting}
11111000000     1984
11111010001     2001
----------- XOR
00000010001       17
\end{lstlisting}

Schritt 2:

\begin{lstlisting}
11111010001     2001
00000010001       17
----------- XOR
11111000000     1984 (= y)
\end{lstlisting}

Schritt 3:

\begin{lstlisting}
11111000000     1984
11111000000     1984
----------- XOR
00000000000        0 (= x)
\end{lstlisting}

\end{frame}

\begin{frame}[fragile]
\frametitle{Operatoren}

\begin{lstlisting}
public class Test {
	public static void main(String[] args) {
		System.out.print("Hell");
		System.out.println("o world");
	}
}
\end{lstlisting}

\begin{itemize}
\item Was schreibt das Programm auf die Konsole?
	\begin{itemize}
	\item \textit{Hello world}
	\item \textit{System.out.print()} schreibt in die Konsole
	\item \textit{System.out.println()} beendet nach dem Schreiben die aktuelle Zeile mit einem Zeilenumbruchzeichen (Betriebssystemabh\"angig)
	\end{itemize}
\end{itemize}

\end{frame}

\begin{frame}[fragile]
\frametitle{Operatoren}

\begin{lstlisting}
public class Increment {
	public static void main(String[] args) {
		int j = 0;
		for (int i = 0; i < 100; i++)
			j = j++;
		System.out.println(j);
	}
}
\end{lstlisting}

\begin{itemize}
\item Was gibt der folgende Code auf der Console aus?
	\begin{itemize}
	\item \textit{0}
	\item Die Inkrementierung durch die ```++''-Operation wird erst nach der Zuweisungsoperation durchgef\"uhrt
	\item Ersetzt man Zeile 5 mit \textit{j = ++j;} funktioniert die gewollte Inkrementierung und der Code schreibt \textit{100} in die Konsole
	\end{itemize}
\end{itemize}

\end{frame}

\end{document}